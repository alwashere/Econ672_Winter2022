\documentclass[12pt]{article}
%\usepackage{indentfirst}
\usepackage{mathptmx}
\usepackage{enumerate}
\usepackage{multirow}
\usepackage{xcolor}
\usepackage[letterpaper, margin=1in]{geometry}


\begin{document}

\title{ECON 672 Winter 2022 Problem Set \#2}
\author{Jon Holder}
%\shortTitle{Problem Set\#2}
%\econClass{ECON 672 Winter 2022}

\newcommand{\question}[1]{\textbf{\noindent Question #1}}
\newcommand{\tabindent}{\hspace{3mm}}

\newcommand{\todo}{\color{red}{\textbf{TODO}}}

\maketitle
% Question 1
\question{1}
    \begin{list}{}{}
        \item The log file shows that the data files is read into Stata. 
        Table \ref{tableQ1} shows the summary statistics in the data set.
        \begin{table}[h!]
            \centering{
\def\sym#1{\ifmmode^{#1}\else\(^{#1}\)\fi}
\begin{tabular}{l*{1}{ccccc}}
\hline\hline
            &\multicolumn{5}{c}{}                                            \\
            &        Mean&          SD&         Min&         Max&     Num Obs\\
\hline
Sex         &    .4568225&    .4981443&           0&           1&       20601\\
Age         &    29.30702&    10.52068&          16&          82&       20601\\
Race        &    1.646037&    .8114741&           1&           4&       20601\\
Earnings 18 months later&    9513.183&    9546.872&           0&       93951&       14192\\
Earnings at initial survey&    2810.803&     4050.78&           0&       64436&       17043\\
Years of educations at initial survey&    11.42821&    1.776717&           7&          18&       20254\\
Childer under 18yr at initial survey&     1.17241&    1.438531&           0&          20&       18659\\
\hline\hline
\end{tabular}
}

            \caption{\textit{Summary Statistics}} 
            \label{tableQ1}
        \end{table}
    \end{list}

% Question 2
\question{2}
    \begin{list}{}{}
        \item See log file.
    \end{list}

% Question 3
\question{3}
    \begin{list}{}{}
        \item See log file.
    \end{list}

% Question 4
\question{4}
    \begin{list}{}{}
        \item Table \ref{tableQ5} model 1 shows estimates of the impact of the 
        experimental assignment to the treatment group without covariates. OLS 
        estimates that the treatment group resulted in \$562.40 increase in 
        average earnings. The estimate is statistically significant at the 1\% 
        level. 
    \end{list}
    \begin{table}[h!]
        \centering{
\def\sym#1{\ifmmode^{#1}\else\(^{#1}\)\fi}
\begin{tabular}{l*{2}{c}}
\hline\hline
            &\multicolumn{1}{c}{(1)}&\multicolumn{1}{c}{(2)}\\
            &\multicolumn{1}{c}{Earnings 18 month later}&\multicolumn{1}{c}{With covariates}\\
\hline
In treatment group&       562.4\sym{**} &       577.8\sym{*}  \\
            &     (184.2)         &     (249.8)         \\
\hline
adj. \(R^{2}\)&       0.001         &       0.174         \\
F           &       9.322         &           .         \\
\(N\)       &       10812         &       10812         \\
\hline\hline
\multicolumn{3}{l}{\footnotesize Standard errors in parentheses}\\
\multicolumn{3}{l}{\footnotesize Estimate with covariates are clustered by site location}\\
\multicolumn{3}{l}{\footnotesize \sym{*} \(p<0.05\), \sym{**} \(p<0.01\), \sym{***} \(p<0.001\)}\\
\end{tabular}
}

        \caption{\textit{OLS Regression Estimates} }
        \label{tableQ5}
    \end{table}

% Question 5
\question{5}
    \begin{list}{}{}
        \item Table \ref{tableQ5} model 2 shows estimates of the impact of the 
        experimental assignment to the treatment group with covariates. OLS 
        estimates that the treatment group resulted in \$577.80 increase in 
        average earnings. The estimate is statistically significant at the 5\% 
        level. This estimate clustered at the site level. The inclusion of the 
        covariates lowered the estimated impact, however the result remains 
        of a similar magnitude and is remains statistically different from zero.
    \end{list}

% Question 6
\question{6}
    \begin{list}{}{}
        \item The estimates from the nearest 10 neighbor propensity score 
        produce similar estimates to the linear regression estimates. There is an 
        increase in estimated effect about \$32 higher in the propensity 
        score matching model.  The propensity score finds matches based on the 
        probability of treatment from the probit model which used the covariates. 
        This is distribution is non-linear. The OLS model linearly estimates a 
        value based on the covariates switching on/off. This would produce some
        differences between the estimates. 
    \end{list}

% Question 7
\question{7}
    \begin{list}{}{}
        \item 
        \begin{table}[h!]
            \centering{
\def\sym#1{\ifmmode^{#1}\else\(^{#1}\)\fi}
\begin{tabular}{l*{2}{c}}
\hline\hline
            &\multicolumn{1}{c}{(1)}&\multicolumn{1}{c}{(2)}\\
            &\multicolumn{1}{c}{Basic Earnings}&\multicolumn{1}{c}{With Covariates}\\
\hline
treatment   &       601.9\sym{***}&       584.4\sym{***}\\
            &     (176.7)         &     (170.5)         \\
\hline
adj. \(R^{2}\)&       0.001         &       0.083         \\
F           &       11.60         &       34.78         \\
\(N\)       &       10812         &       10812         \\
\hline\hline
\multicolumn{3}{l}{\footnotesize Standard errors in parentheses}\\
\multicolumn{3}{l}{\footnotesize \sym{*} \(p<0.05\), \sym{**} \(p<0.01\), \sym{***} \(p<0.001\)}\\
\end{tabular}
}

            \caption{\textit{Difference-in-Difference Estimates} }
            \label{tableQ7}
        \end{table}
    \end{list}

% Question 8
\question{8}
\begin{list}{}{}
    \item Table \ref{tableQ7} shows a difference in means estimate between the 
    JTPA control group and the JTPA treatment. The DID models estimate that 
    enrollment in the JTPA treatment resulted in an approximately \$600 increase 
    in average earnings when compared to control group. Adding covariates to 
    the basic DID model reduced the impact of the program towards the basic OLS 
    estimate with covariates. Both DID estimates are significant at the 1\% 
    level. The results of the DID are in the neighborhood of 
    the propensity score matching results. The propensity score ATT estimate is 
    \$594.16 with the basic DID estimate of \$601.90 and DID estimate with 
    covariates of \$584.40.
\end{list}

% Question 9
\question{9}
\begin{list}{}{}
    \item See log file.
\end{list}

% Question 10
\question{10}
\begin{list}{}{}
    \item See log file.
\end{list}

% Question 11
\question{11}
\begin{list}{}{}
    \item See Table \ref{tableQ11}.
    \begin{table}[h!]
        \centering{
\def\sym#1{\ifmmode^{#1}\else\(^{#1}\)\fi}
\begin{tabular}{l*{1}{c}}
\hline\hline
            &\multicolumn{1}{c}{(1)}\\
            &\multicolumn{1}{c}{Window}\\
\hline
At or below cutoff&      1126.4         \\
            &     (962.6)         \\
\hline
adj. \(R^{2}\)&       0.001         \\
F           &       1.369         \\
\(N\)       &         329         \\
\hline\hline
\multicolumn{2}{l}{\footnotesize Standard errors in parentheses}\\
\multicolumn{2}{l}{\footnotesize Cutoff at \$2,650}\\
\multicolumn{2}{l}{\footnotesize Window is -/+ \$500}\\
\multicolumn{2}{l}{\footnotesize \sym{*} \(p<0.05\), \sym{**} \(p<0.01\), \sym{***} \(p<0.001\)}\\
\end{tabular}
}

        \caption{\textit{Regression Discontinuity via Difference in Means 
                        Estimates} }
        \label{tableQ11}
    \end{table}
\end{list}

% Question 12
\question{12}
\begin{list}{}{}
    \item The estimate shown in Table \ref{tableQ11} estimate the impact of 
    receiving treatment on earnings 18 months after the treatment for 
    participants with prior earnings in the -/+ \$500 window around \$2,650. 
    The model estimates that receiving the treatment causes a \$1,126.40 
    increase in earnings although the estimate is not statistically different 
    from zero.
\end{list}

% Question 13
\question{13}
\begin{list}{}{}
    \item The result of Table \ref{tableQ11} is not surprising because the 
    treatment in this setting is hypothetical.  The estimate indicates that this 
    treatment had no impact. 
\end{list}

% Question 14
\question{14}
\begin{list}{}{}
    \item See Table \ref{tableQ14}.
    \begin{table}[h!]
        \centering{
\def\sym#1{\ifmmode^{#1}\else\(^{#1}\)\fi}
\begin{tabular}{l*{2}{c}}
\hline\hline
            &\multicolumn{1}{c}{(1)}&\multicolumn{1}{c}{(2)}\\
            &\multicolumn{1}{c}{All Data}&\multicolumn{1}{c}{Window}\\
\hline
At or below cutoff&     -3217.6\sym{***}&     21449.5         \\
            &     (597.8)         &   (20043.3)         \\
[1em]
Earnings 1 year prior&       0.596\sym{***}&       11.37\sym{*}  \\
            &    (0.0808)         &     (4.647)         \\
[1em]
$\beta$Interaction&       1.518\sym{***}&      -5.884         \\
            &     (0.236)         &     (7.711)         \\
\hline
adj. \(R^{2}\)&       0.127         &       0.013         \\
F           &       154.4         &       3.725         \\
\(N\)       &        3757         &         329         \\
\hline\hline
\multicolumn{3}{l}{\footnotesize Standard errors in parentheses}\\
\multicolumn{3}{l}{\footnotesize Cutoff at \$2,650}\\
\multicolumn{3}{l}{\footnotesize Window is -/+ \$500}\\
\multicolumn{3}{l}{\footnotesize \sym{*} \(p<0.05\), \sym{**} \(p<0.01\), \sym{***} \(p<0.001\)}\\
\end{tabular}
}
. 
        \caption{\textit{Regression Discontinuity Estimates} }
        \label{tableQ14}
    \end{table}
\end{list}

% Question 15
\question{15}
\begin{list}{}{}
    \item Table \ref{tableQ14} shows regression discontinuity estimates that 
    allow both the y-intercept and slope on both sides of the discontinuity to 
    vary.  The estimates using the full data set show a statistically and 
    economically significant result that the treatment reduces future earnings. 
    However, the economic effect becomes substantially positive and statistically 
    indistinguishable from zero when reducing the window to a sample that is 
    more localized to the treatment.

    The results of both estimates are not surprising.  The full dataset shows the 
    importance of making sure the comparison groups are comparable.  The 
    windowed estimate shows the impact of the psuedotreatment was not different 
    from zero.  

    I prefer the windowed estimate.  It seems more reasonable that participants 
    in that narrow pretreatment income band are very comparable.  I expect that 
    the differences between the full treatment group and full control would 
    prevent a valid comparison for this type of treatment.
\end{list}

\end{document}