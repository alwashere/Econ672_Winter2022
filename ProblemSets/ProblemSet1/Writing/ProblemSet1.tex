\documentclass[12pt]{article}
%\usepackage{indentfirst}
\usepackage{mathptmx}
\usepackage{enumerate}
\usepackage{multirow}
\usepackage{xcolor}
\usepackage[letterpaper, margin=1in]{geometry}

\usepackage[authordate,backend=bibtex]{biblatex-chicago}
\addbibresource{../Articles/reference.bib}

\begin{document}

\title{ECON 672 Winter 2022 Problem Set \#1}
\author{Jon Holder}
%\shortTitle{Problem Set\#2}
%\econClass{ECON 672 Winter 2022}

\newcommand{\question}[1]{\textbf{\noindent Question #1}}
\newcommand{\tabindent}{\hspace{3mm}}

\newcommand{\todo}{\color{red}{\textbf{TODO}}}

\maketitle
% Question 1
\question{1}
    \begin{enumerate}[\indent a.]
        \item \textcite{Persico2004} are examining the mechanism that 
        influences the relationship between height and earnings gap. They note 
        that previous studies have evaluated the influence of current height.  
        They take advantage of the fact that height changes throughout youth 
        before becoming relatively constant in adulthood. 
        \citeauthor{Persico2004} find that height at 16 is a predictor of 
        future earnings controlling for vocation, family attributes and adult 
        height.
        
        \item Two data sets are evaluated: 1) Britain's National Child 
        Development Survey (NCDS) and the US's National Longitudinal Survey of 
        Youth (NLSY) 1979 cohort.

        \item Table 1 shows the raw heights, change in heights and their 
        distributions for white men in the two data sets. 
        \citeauthor{Persico2004} include this data because it is the key 
        parameter of interest.  The characteristics and distributions of these 
        heights provide important context for the conclusion of the authors.

        \item Table 2 shows various statistics for sample.  It includes 
        traditional determinants of wages include individual and family 
        educational attainment, martial status, and number of siblings. Finally, 
        Table 2 contains the ln(wages) which is the dependent variable.

        \item Table 3 shows various OLS estimates for effect on ln(wages) 
        using the controls from Table 3.  The estimates show a positive return
        to ln(wages) for full-time white male workers based on reported height 
        during youth. The reduced form specification shows a positive and 
        statistically significant positive return to wages based on adult 
        height. However, the return due to adult height diminishes and is 
        no longer statistically significant when youth height is added to the 
        specification.
                
        \item \citeauthor{Persico2004} explore the relationship between height 
        at various stages of development age 7, 11 and 16 in addition to adult 
        height.  The table shows that the height premium is only statistically 
        economically significant at age 16 when all heights and family 
        characteristics are included.
        
        \item The primary conclusion of the paper is that height when a 
        teenager is the primary contributor to adult wage differential rather 
        than adult height.  The authors tested for height as indicator of 
        several factors including health, availability of external resources, 
        and intergenerational factors.  There appears to be an identified 
        mechanism that associated with height that produces a future wage 
        premium. The authors suggest this premium is gained through greater 
        social participation of the taller respondents. This premium is present 
        independent of career choices. 
        
        Additionally, the wage premium is high enough that receiving growth 
        hormone treatments to increase teen height has a positive return.
    \end{enumerate}

% Question 2
\question{2}
    \begin{enumerate}[\indent a.]
        \item[b.]
            \begin{enumerate}[(i)]
                \item NLSY contained 3 subsamples: a cross-sectional sample that 
                represented noninstitutionalized civilian population, a 
                sample that oversampled civilian Latino, black and other 
                economically disadvantage civilians, and a sample representing 
                the population that was in the military.  After the 1984, most 
                members of the military where ineligible for interview so 201 
                members of that cohort were randomly selected to remain. After 
                the 1990 interview, the economically disadvantaged white 
                population was removed from the sample as noted in the analysis 
                by \textcite{Persico2004}.

                \item The survey collected information about the respondents 
                family background and family formation. They collected 
                information about the respondents education background including 
                high school, college, and/or vocational training as well as a 
                standardized test of general ability, ASVAB. Finally, the survey 
                collected information about labor participation, government 
                program participation, and income and assets.
            \end{enumerate}
        \item[f.] Height data is available in 1981, 1982,1985, 2006, 2008, 2010 
        2012, 2014. 2016, 2018.
    \end{enumerate}

\question{4}
    \begin{enumerate}[\indent a.]
        \item There are 12,686 observations in the dta file.

        \item There in are 6,403 (50.47\%) males and 6,283 (49.53\%) females in 
        the 1979 sample.
    \end{enumerate}

\question{5}
\begin{itemize}
    \item[] Restrict sample to non-poor white men.
\end{itemize}

\question{6}
\begin{enumerate}[\indent a.]
    \item Generate \textbf{height81}
    \begin{enumerate}[(i)]
        \item HEALTH\_HEIGHT\_1981 is encoded as height in feet and inches 
        displayed as a 3 digit integer value.

        \item -5 response means that the person was not interviewed.
        
        \item Responses below 0 are error codes.  Additionally height 
        cannot be negative.
        
        \item A response of "510" means that respondent is 5 feet and 10 
        inches tall.
        
        \item \textit{Generate Stata variable}
        
        \item The mean height for white men in 1981 is 70.34 inches or about 
        5ft 10.3in.
    \end{enumerate}
    
    \item Generate \textbf{height85}
    \begin{enumerate}[(i)]
        \item \textit{Generate Stata variable}
        
        \item The mean height of a white man in the 1985 sample is 70.64 inches 
        or 5ft 10.65inches.

        \item Error coded observations were dropped.  Values less than 0 are 
        error coded.

        \item The mean change in height between 1985 and 1981 is .2889 inches.
    \end{enumerate}
    
    \item Generate \textbf{age96}
    \begin{enumerate}[(i)]
        \item FAM\_1B\_1979 is the respondent's age in 1979.
        
        \item \textit{Generate Stata variable}
    \end{enumerate} 

    \item Generate \textbf{income}
    \begin{enumerate}[(i)]
        \item Income in truncated for the top 2\%. Negative values are error 
        codes but 0 is valid value.

        \item \textit{Generate Stata Variable}
        
        \item There are 2,094 valid observations for income.
        
        \item  The mean of the valid income is \$34,912.86.
    \end{enumerate}

    \item Generate \textbf{hours}
    \begin{enumerate}[(i)]
        \item Negative values are error codes but 0 is valid value.
        
        \item \textit{Generate Stata Variable}
        
        \item There are 2,139 valid observations for hours in the dataset.
        
        \item The mean hours worked is 2,222.4.
        
        \item Generate \textit{fulltime} variable 

        \item 90.04\% of the remaining respondents worked fulltime.
    \end{enumerate} 

    \item Generate \textbf{lnWage}
    \begin{enumerate}[(i)]
        \item \textit{Generate Stata Variable}
        
        \item There are 1,973 observations that have non-missing lnWage values.
        
        \item The remaining sample size with non-missing values is 1,979.
    \end{enumerate}

    \item Generate \textbf{educ}.  This is the highest level of education 
    reported from the 1979,1981, 1985, and 1996 surveys.
    \begin{enumerate}
        \item Missing values are negative.
        
        \item \textit{Generate Stata Variable}

        \item The mean number years completed is 12.2 yrs.  This corresponds to 
        a little more than a high school
    \end{enumerate}

    \item Generate \textbf{everMarried}
    \begin{enumerate}
        \item Values less than 0 are error codes. Values 1 or higher indicate 
        the respondent is currently or was previously married and is now either 
        separated, divorced or widowed.

        \item \textit{Generate Stata Variable}
        
        \item About 78.48\% of the sample have been married.
    \end{enumerate}    
    
    \item Generate \textbf{momSchool}
    \begin{enumerate}[(i)]
        \item Error codes are negative.  A value of 95 means the respondent has 
        an ungraded highest level of education.  However, no respondents have 
        this value.

        \item \textit{Generate Stata Variable}
        
        \item The mean level of the mother's education for the sample is 11.9 
        years.  
    \end{enumerate}

    \item Generate \textbf{momSkilled}
    \begin{enumerate}[(i)]
        \item This contains classification codes.  Codes 1-245 are considered 
        professional/managerial. Negative values are error codes.

        \item \textit{Generate Stata Variable}
        
        \item The share of mothers in skilled occupations in 1979 is 19.7\%.
    \end{enumerate}

    \item Generate \textbf{dadSchool}. Generate \textbf{dadSkilled}
    \begin{enumerate}[(i)]
        \item The mean level of school for dads is 12.2 years.
        
        \item The share of dads in skilled occupations in 1979 is 32.2\%
    \end{enumerate}

    \item Generate \textbf{siblings}
    
    \item Generate \textbf{finalSample}
    \begin{enumerate}[(i)]
        \item \textit{Generate Stata Variable}
        
        \item The final sample size is 910 responses.
    \end{enumerate}

\end{enumerate}

\question{7}
\begin{table}[h!]
    \begin{tabular}{*{7}{|c}|} 
        \hline 
         & Mean & Median & Standard Deviation & $25^{th}$ Percentile &
        $75^{th}$ Percentile & Observations \\
        \hline
        \multicolumn{7}{|c|}{A. United States NLSY Entire Sample} \\
        \hline
        Height 1981 & 70.34 & 71 & 2.82 & 69 & 72 & 2900 \\
        \hline
        Height 1985 & 70.64 & 71 & 2.78 & 69 & 72 & 2343 \\
        \hline
        $\Delta$1981-1985 & -0.29 & 0 & 1.44 & -1 & 0 & 2281 \\
        \hline
        \multicolumn{7}{|c|}{B. Final Estimation Sample} \\
        \hline
        Height 1981 & 70.33 & 71 & 2.90 & 69 & 72 & 860 \\
        \hline
        Height 1985 & 70.70 & 71 & 2.75 & 69 & 72 & 860 \\
        \hline
        $\Delta$1981-1985 & -0.37 & 0 & 1.49 & -1 & 0 & 860 \\
        \hline
        \multicolumn{7}{l}{\footnotesize Final sample is full-time employed 
        white males who had non-missing data in the control variables.} \\
    \end{tabular}
    \caption{\textit{Replication of \textcite{Persico2004} Table 1}}
\end{table}

\clearpage
\def\sym#1{\ifmmode^{#1}\else\(^{#1}\)\fi}
\question{8}
\begin{table}[h!]
    \begin{tabular}{|l | c | c | c |} 
        \hline 
         & Adult Height  & 
           Adult Height  & 
            \\
         & Median or Below  & 
           Above Median & 
            Difference \\
         & (1) & 
           (2) & 
           (3) \\
        \hline
        Adult Characteristics & & &\\
        \hline
        \tabindent 1981 Height & 68.60  & 72.52  & 3.92\sym{***} \\
                            & (2.31) & (1.92) & (0.14)\\
        \hline
        \tabindent 1985 Height & 68.81  & 73.09  & 4.28\sym{***} \\
                           & (2.04) & (1.27) &  (0.11) \\
        \hline
        \tabindent Age     & 34.44  & 34.48  & 0.05 \\
                           & (2.24) & (2.32) &  (0.16)       \\
        \hline
        \tabindent ln(wage/hour)     & 2.61   & 2.71  & 0.010\sym{*} \\
                                     & (0.60) & (0.61) &  (0.04)       \\
        \hline
        \tabindent Ever Married(\%)  & 86.67  & 89.74 & 0.03 \\
                                     & (34.02) & (30.39) & (0.02) \\
        \hline
        Family Background & & & \\
        \hline
        \tabindent Mother's years of schooling & 12.26 & 12.49 & 0.23\\
                                               & (2.45) & (2.09) & (0.15) \\
        \hline
        \tabindent Mother skilled$\backslash$professional (\%) & 
                                                    18.75 &  21.57 & 2.83 \\
                                                 & (39.07) & (41.19) & (2.76) \\
        \hline
        \tabindent Father's years of schooling & 12.50 & 12.86 & 0.36 \\
                                               & (3.22) & (3.09) & (0.22) \\
        \hline
        \tabindent Father skilled$\backslash$professional (\%) &
                                                 31.88 & 35.00 & 3.12 \\
                                              & (46.65) & (47.76) & (3.25) \\
        \hline
        Observations & 580 & 380 & 860 \\
        \hline\hline   
        \multicolumn{4}{l}{\footnotesize Standard errors in parentheses} \\
        \multicolumn{4}{l}{\footnotesize Full-time employed white males} \\
	\multicolumn{4}{l}{\footnotesize \sym{*} \(p<0.05\), \sym{**} \(p<0.01\), 
                        \sym{***} \(p<0.001\)}\\
    \end{tabular}
    \caption{\textit{Replication of \textcite{Persico2004} Table 2.}}
\end{table}

\clearpage

\question{9}
\begin{table}[h!]
	{
\def\sym#1{\ifmmode^{#1}\else\(^{#1}\)\fi}
\begin{tabular}{l*{4}{c}}
\hline\hline
            &\multicolumn{1}{c}{(1)}&\multicolumn{1}{c}{(2)}&\multicolumn{1}{c}{(3)}&\multicolumn{1}{c}{(4)}\\
            &\multicolumn{1}{c}{lnWage}&\multicolumn{1}{c}{lnWage}&\multicolumn{1}{c}{lnWage}&\multicolumn{1}{c}{lnWage}\\
\hline
Adult Height (inches)&      0.0303\sym{***}&      0.0230\sym{**} &      0.0136         &     0.00694         \\
            &   (0.00764)         &   (0.00747)         &    (0.0136)         &    (0.0124)         \\
[1em]
Youth height (inches)&                     &                     &      0.0184         &      0.0178         \\
            &                     &                     &    (0.0123)         &    (0.0113)         \\
[1em]
Age         &      0.0271\sym{**} &      0.0279\sym{**} &      0.0240\sym{**} &      0.0249\sym{**} \\
            &   (0.00870)         &   (0.00849)         &   (0.00888)         &   (0.00869)         \\
[1em]
Mother's years of schooling&                     &      0.0129         &                     &      0.0117         \\
            &                     &    (0.0125)         &                     &    (0.0124)         \\
[1em]
Mother Skilled/Professional&                     &     -0.0225         &                     &     -0.0200         \\
            &                     &    (0.0575)         &                     &    (0.0575)         \\
[1em]
Father's years of schooling&                     &      0.0285\sym{**} &                     &      0.0289\sym{**} \\
            &                     &   (0.00887)         &                     &   (0.00885)         \\
[1em]
Father Skilled/Professional&                     &       0.132\sym{**} &                     &       0.133\sym{**} \\
            &                     &    (0.0478)         &                     &    (0.0478)         \\
[1em]
Number of siblings&                     &     -0.0146         &                     &     -0.0147         \\
            &                     &    (0.0113)         &                     &    (0.0112)         \\
\hline
adj. \(R^{2}\)&       0.027         &       0.086         &       0.028         &       0.087         \\
F           &       12.33         &       12.57         &       8.995         &       11.26         \\
\(N\)       &         860         &         860         &         860         &         860         \\
\hline\hline
\multicolumn{5}{l}{\footnotesize Standard errors in parentheses}\\
\multicolumn{5}{l}{\footnotesize \sym{*} \(p<0.05\), \sym{**} \(p<0.01\), \sym{***} \(p<0.001\)}\\
\end{tabular}
}

    \caption{\textit{Replication of \textcite{Persico2004} Table 3 for NLSY}} 
\end{table}
\begin{enumerate}[a.]
    \item The reduced form specification (1) shows similar results to 
    \textcite{Persico2004}.  There statistically and economically significant 
    returns to wages for taller men.  An additional inch will increase wages 
    by 3\% holding all else equal. Including family characteristics as 
    controls in regression (2) reduces the return to adult height to 2.3\%. 
    It remains economically and statistically significant. Considering teen 
    height in specification (3) further reduces the returns to adult height. 
    Finally, considering teen height and family characteristics in specification 
    (4) reduces the returns to adult height to an economically small value 
    while return to youth height remains about the same as specification (3). 
    
    Unlike \textcite{Persico2004}, the returns to teen height are statistically 
    indistinguishable from zero in specification (3) and (4). This 
    maybe to the different samples evaluated.  The authors used a sample of 
    2,063 responses while I only had 860 valid responses. The general trends 
    in the results we of the same direction.
\end{enumerate}

\printbibliography 

\end{document}